%-------------------------%
%-----Document Setup------%
%-------------------------%
\documentclass[ebook,11pt,oneside,openany]{memoir}
\usepackage[utf8x]{inputenc}
\usepackage[english]{babel}
\usepackage[spanish]{babel}
\usepackage{savetrees}
\usepackage{verbatim}
\linespread{1.6}
\setlength{\footskip}{20pt}

%-------------------------%
%------Document Code------%
%-------------------------%
\newcommand{\thought}[1]{\textit{#1}}

\newcommand{\scenechange}{
  \par
  \vspace{\baselineskip}
  \par
\noindent}
%Creates a line break for a change of scene

\newcommand{\majorchange}{
  \par
  \vspace{\baselineskip}
  \hfill
  \textasteriskcentered
  \hfill
  \vspace{\baselineskip}
\noindent}
%creates a major line break, split by an asterisk for scene changes at the end of a page of where a sense of a major change is required. 

%-------------------------%
%------Main Document------%
%-------------------------%
\begin{document}
\title{Barrios}
\author{Pablo Rodriguez Ros}
\date{}
\maketitle

\section*{}
De esas entidades de organización suburbana que persisten y dignifican.

Cartagena 1990 - 2010
Educación pública preparación para la realidad de tu alrededor. Sin entender ese alrededor como un país, sino como jn barrio o como mucho una ciudad. El objetivo de la educación debe de ser poder comprender y tolerar la diversidad que envuelve la vida de una persona, pues los conocimientos nunca son tan importnsges. De hecho, los conocimientos serán corregidos, refundados y vlteados una vez se llegue, si se llega, a la universidad. Y el progreso hará un poco más de lo mismo. Lo importante de la educación primaria y secundaria no es aprender mucho, es aprender bien y conocer tu entorno.

→ Ser de izquierdas porque los hijos no tienen la culpa de las malas decisiones de sus padres. Al menos para unos mínimos, la formación de calidad sea al nivel que sea debe de estar garantizada si se demuestra valía y actitud.

Murcia 2010 - 2012

Los bocadillos del guarro.

Cádiz 2011

Es que eres del Norte. Vivir delante de la playa.

Aberdeen 2013

Problemas con la inmigración:

Cuando me gritaron por la calle fucking inmigrants. Cuando me paró la policia en la calle para peidrme el pasaprote y preguntarme si tenía la nacionalidad espaóla o era de irán o de oriente medio. Cuando en un festival de música me metieron en una caseta al entrar y me interrgoraron también para ver si era inmigrante, llevaba el DNI pero solo les valía el pasaporte (UK aún era Europa y todavía teníamos que llevar pasaporte. Qué maravilla de “Unión” Europea).

Palma de Mallorca 2013 - 2014

El turismo. Naum son roca. Ovejas en el monte y zarigüeyas, degradación ambiental y especies invasoras en islas 

Barcelona 2014 - 2019

Nacionalismo.

Nunca me he sentido una persona nacionalista. Nunca he tenido chovinismo hacia un país, siempre me ha parecido absurdo a la par que imposible. Cómo voy a a apreciar un país que esta lleno de hijosdephta solamente porque comparto con ellos el nivel de organización estatal? Lo mismo para comunidades autónomas e incluso ciudades. Solo reconozco el barrio como entidad en la que mirarse, tal vez reconocerse, y mostrar orgullo.

Eso no significa que no amar signifique odiar. No, no es eso. No odio a mi país, no odio a mi comunidad, ni a mi ciudad; solo reconozco en ellas la injusticia y las calamidades más grandes que pueden hacerse.

No por tanto mi lucha la nacionalista. Respeto quien la tenga, no me parece algo per se malo. Siempre que no se caiga en el chovinismo xenófobo (tan frecuente en Europa por desgracia). Las diferencias culturales (si es que eso existe y es medible) son diferencias, a secas. Que algo sea diferente no implica superioridad, ni inferioridad. El talento, las capacidades y, en última instancia, las buenas personas están homogéneamente distribuidas. No entienden de nacionalidades. O al menos eso es lo que yo creo 

Québec 2016

Frío, niños que juegan vestidos con mil capas. Centros deportivos cubiertos gratuitos. Poutine.

Zurich 2017&2018

El estado del bienestar. El capitalismo. La desigualdad. El mundo feliz basado en la explotación de otros. 

Expulsión refugiado tren en Austria. Me pidieron también a mí el pasaporte.

San Diego 2019 - 2020

La naturaleza ha cambiado sin cambiar, ha sido dañada y restituida, y comprendida solo en parte. Los límites de la misma son difusos, al igual que lo es su propia definición. La naturaleza es ubicua, no hace falta buscarla espacialmente en lugares lejanos, basta con mirar aquí y ahora. La exploración de los siglos pasados ha dado paso a la ciencia. La heroicidad arcaica de los exploradores ha dado paso a la inervación meticulosa de los fenómenos naturales. Comprender los procesos que rigen la naturaleza comprendida entre la materia inerte y la viva, a veces indistinguibles. Pues la propia historia natural nos dice que todo esta conectado, todo se intercambia. Es así como podemos entender, por ejemplo, nuestro planeta como un conjunto de varios compartimentos, los cuales se mueven en las cuatro dimensiones del espacio y los límites, por tanto, a veces no existen aqui y ahora. Tal vez, hablar de compartimentos induzca a error, pues la mente humana tienen a clasificarlo todo en cajas y parece que es la única manera que tienen muchos de comprender el mundo. Estos compartimentos, también llamados esferas (biosfera, hidrosfera... Etc), no son más que la definición de un conjunto de materia que se rige por unos determinados procesos de intercambio de materia y energía. Una clasificación que, de nuevo, no es precisa, pero es la mejor que tenemos 

Unas esferas se pegan a otras, se mezclan, se arañan, desprenden, intercambian o viven en simbiosis. La arcilla llena de agua, con lombrices que se pasean por ella emitiendo gases: geosfera, hidrosfera, biosfera y atmósfera viviendo juntas. ¿Dónde está el limite entre ellas? Donde acabas una y empieza la otra. La naturaleza es pues la respuesta de fuerzas aún incomprensibles. El eterno latir del eco lejano del big bang.

El kosmos de Humboldt, el Microcosmos de Margulis o la Gaia de Lovelock. Unificadores de la grandilocuencia holisitica naturaleza. 

"No afecta a la naturaleza", " la naturaleza se recupera". Expresiones tecnocentristas de científicos de mente compartimentalizada. Ególatras de saber restringido, que obvian lo que desconocen. Porque la humildad de reconocer la ignorancia, parece que en este siglo ha dado paso a la prepotencia del experto en algo atómico del infinito universo del saber. Tal vez, el estudio de la naturaleza esté llegando a su fin, abocandonos al estudio de la naturaleza antropocentrista obviando el valor inherente de la misma y de ambos, pues son la misma cosa. Imperceptible es nuestro tiempo formando parte viva y consciente de la naturaleza para ella misma. Tenemos poco tiempo para comprenderla, para comprendernos, y dejar así una muesca un saber algo que flota y se transmita. Pero es algo que ya se sabe o no se sabe, que es lo mismo. El ser humano descubre cosas que ya existen, cosas que funcionan y que son indiferentes de que sepamos como funcionan o no. Descubrimos para nosotros, para nuestra gloria y, en ultima instancia, para nuestra supervivencia y casi siempre llegamos tarde. 




\end{document}